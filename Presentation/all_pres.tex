%\documentclass[a4paper]{article}
%\usepackage{beamerarticle}
\documentclass[ignorenonframetext,unicode,handout, 9pt]{beamer}

\usepackage[utf8]{inputenc}
\usepackage[T2A]{fontenc}
\usepackage[english,russian]{babel}
\usepackage{amsmath}
\usepackage{amsfonts}
\usepackage{amssymb}
\usepackage{graphicx,pgf}
\usepackage{multimedia}
\usepackage{ragged2e}
\justifying
%\usepackage{hyperref}

%\usetheme{Rochester}  %тема без навигации
\usetheme{Montpellier} %тема с навигацией в виде дерева
%\usetheme{Berkeley} %тема с оглавлением на полях
%\usetheme{Berlin} %тема с навигацией в виде мини-слайдов
\mode<presentation>{
%\usetheme{Copenhagen} %тема с таблицей разделов и подразделов

\graphicspath{{pictures/}}
\DeclareGraphicsExtensions{.pdf,.png,.jpg}



\numberwithin{equation}{section}
%\numberwithin{equation}{subsection}
%\renewcommand{\theequation}{\arabic{equation}}
\renewcommand{\theequation}{\arabic{section}.\arabic{equation}}

%\useinnertheme{circles}   %внутренняя тема
%\useoutertheme{smoothbars}   %внешняя тема
\usecolortheme{beaver}     %цветовая схема
\usefonttheme{serif}    %шрифты
}
% \\ {} \and \\  \\[3mm]
\title[Периодические решения уравнения колебания балки с жестко закрепленным концом]{Периодические решения уравнения колебания балки с жестко закрепленным концом}
\author[Выполнил: М.Д Зиновьев]{Выполнил: М.Д Зиновьев \and  \\ Науч. руководитель: д.ф - м.н., профессор, Рудаков И.А.}
\institute{МГТУ имени Н.Э.Баумана (национальный исследовательский университет) \and Кафедра ФН-2(прикладная математика)}
\date{\today}
\titlegraphic{\includegraphics[width=2cm]{logo.png}}
%\setbeamersize{text margin left=4cm}

\begin{document}


\begin{frame}[plain]
\maketitle
\end{frame}

\section{Постановка задачи 1}

\begin{frame}{Постановка задачи 1.}
\setbeamercovered{transparent}

Рассматривается уравнение Эйлера---Бернулли

\begin{equation}\label{equationkv}
  u_{tt} + u_{xxxx} - au_{xx} = g(x, t, u) + f(x,t), \; x \in (0,\pi), \; t \in \textbf{R}
\end{equation}

с однородными граничными условиями на отрезке

\begin{equation}\label{ickv1}
  u(0,t) = u_{xx}(0,t) = 0, \; t \in \textbf{R}
\end{equation}

\begin{equation}\label{ickv2}
  u(\pi,t) = u_{x}(\pi,t) = 0, \; t \in \textbf{R}
\end{equation}

и с условием периодичности по времени

\begin{equation}\label{periodkv}
 u(x,t + T) = u(x,t), \; x \in (0,\pi), \; t \in \textbf{R}.
\end{equation}

\end{frame}

%%%%%%%%%%%%%%%%%%%%%%%%%%%%%%%%%%%%%%%%%%%%%%%%%%%%%%%%%%%%%%%%%%%%%%%%%%%%%%%%%%%%%%

\section{Линейное уравнение}
\subsection{Линейная задача}

\begin{frame}{Линейная задача.}

Рассмотрим линеаризованное уравнение:

\begin{equation}\label{equation}
  u_{tt} + u_{xxxx} - au_{xx} =  f(x,t), \; x \in (0,\pi), \; t \in \textbf{R}
\end{equation}

с однородными граничными условиями на отрезке

\begin{equation}\label{ic1}
  u(0,t) = u_{xx}(0,t) = 0, \; t \in \textbf{R}
  \end{equation}

\begin{equation}\label{ic2}
  u(\pi,t) = u_{x}(\pi,t) = 0, \; t \in \textbf{R}
\end{equation}

и с условием периодичности по времени

\begin{equation}\label{period}
 u(x,t + T) = u(x,t), \; x \in (0,\pi) , \; t \in \textbf{R}.
\end{equation}

\end{frame}

%%%%%%%%%%%%%%%%%%%%%%%%%%%%%%%%%%%%%%%%%%%%%%%%%%%%%%%%%%%%%%%%%%%%%%%%%%%%%%%%%%%%%%

\begin{frame}

Задача Штурма---Лиувилля на поиск собственных  значений оператора уравнения (\ref{equation})

\begin{equation}\label{task}
  X'''' - a X'' = \lambda X, x \in (0,\pi),\; a > 0
\end{equation}

с граничными условиями

\begin{gather*}
  X(0) = X''(0) =0  \label{bc1}, \\
  X(\pi) = X'(\pi) = 0 \label{bc2}.
\end{gather*}

Трансцендентное уравнение:

\begin{multline}\label{313}
\th\bigg(\pi \sqrt{\frac{1}{2}(a+\sqrt{a^{2}+4\lambda})}\bigg) \cdot \sqrt{\frac{1}{2}(\sqrt{a^{2}+4\lambda-a})} - \\
\tg\bigg(\pi \sqrt{\frac{1}{2}(\sqrt{a^{2}+4\lambda}-a)}\bigg) \cdot \sqrt{\frac{1}{2}(a+\sqrt{a^{2}+4\lambda})} = 0
\end{multline}

\end{frame}

%%%%%%%%%%%%%%%%%%%%%%%%%%%%%%%%%%%%%%%%%%%%%%%%%%%%%%%%%%%%%%%%%%%%%%%%%%%%%%%%%%%%%%

\begin{frame}

Собственные функции можно представить ввиде:

\begin{equation}
X_n = C_n (\sin(y_n x) - \frac{\sin(y_n \pi)}{\sh(z_n \pi)} \sh(z_n x)).
\end{equation}

Из условия нормировки константа $C_n$ равна
\begin{equation}\label{Cn}
C_n = \sqrt{\frac{2}{\pi}} + \beta_n, \; \lim_{n \rightarrow \infty}\beta_n = 0.
\end{equation}

Произведя локализацию собственных значений, получим
\begin{equation}\label{ln}
  \lambda_n = \bigg(n + \frac{1}{4} - \theta_n\bigg)^4 + a\bigg(n + \frac{1}{4} - \theta_n\bigg)^2,\;\theta_n \in \bigg(0;\frac{1}{4}\bigg)
\end{equation}

Также имеют место оценки:
\begin{equation}\label{thet}
 \frac{a}{16(a + 1)} \cdot \frac{1}{n^2} < \theta_n < \frac{a+ 2}{2 \pi} \cdot \frac{1}{n^2},\; n \in \textbf{N}
\end{equation}
\begin{equation}\label{Xnk}
  |X_n^{(k)}| \leqslant A_k n^k, A_k = const.
\end{equation}


\end{frame}

%%%%%%%%%%%%%%%%%%%%%%%%%%%%%%%%%%%%%%%%%%%%%%%%%%%%%%%%%%%%%%%%%%%%%%%%%%%%%%%%%%%%%%

\subsection{Собственные функции оператора $A$ и их свойства}

\begin{frame}{Собственные функции оператора $A$.}

Введем следующие условия на период и коэффициент $a$
\begin{equation}\label{tabc}
  T = 2\pi\frac{b}{c},\; b,c \in \textbf{N}, \; (b,c) = 1,
\end{equation}
\begin{equation}\label{a}
  a > 0, \; (a + 1/8)b \notin \textbf{N}.
\end{equation}

Введем обозначение: $\Omega = [0;\pi] \times \textbf{R}/ (T \textbf{Z})$. Решение исходной задачи будет представлено в виде ряда Фурье по следующей системе функций в $L_2(\Omega)$

\[\left\{\begin{array}{lr}\label{fwef}\frac{1}{T} X_n,  e_{nm}^c, e_{nm}^s\end{array}\right\}.\]

\begin{equation}
  e_{nm}^c = \sqrt{\frac{2}{T}}X_n \cos\bigg(\frac{2\pi}{T}m t\bigg), \;\; e_{nm}^s = \sqrt{\frac{2}{T}}X_n \sin\bigg(\frac{2\pi}{T}m t\bigg) \label{enm}
\end{equation}

Функции (\ref{enm}) являются собственными функциями дифференциального оператора $A = \frac{\partial^2}{\partial t^2} + \frac{\partial^4}{\partial x^4} - a \frac{\partial^2}{\partial x^2}$ исходной задачи с собственными значениями

 \begin{equation}\label{munm}
   \mu_{nm} = \lambda_n - \bigg(\frac{c}{b}m\bigg)^2,\; n \in N, \; m \in Z_+,
 \end{equation}

\end{frame}

%%%%%%%%%%%%%%%%%%%%%%%%%%%%%%%%%%%%%%%%%%%%%%%%%%%%%%%%%%%%%%%%%%%%%%%%%%%%%%%%%%%%%%

\subsection{Лемма}

\begin{frame}{Лемма для линейного уравнения.}
\textbf{Лемма}. Пусть выполнены условия (\ref{tabc}), (\ref{a}). Тогда оператор $A^{-1} : R(A) \rightarrow D(A)$ является вполне непрерывным, обобщенное решение $u$ исходной задачи существует и единственно на множестве $D(A) \cap R(A)$ и имеют место включения: \\
если $f \in L_2(\Omega)\cap R(A)$, то
\begin{equation}\label{1}
  u = A^{-1}f \in H_1(\Omega)\cap C(\Omega), \; u_x \in C(\Omega),
\end{equation}
если $f \in  H_1(\Omega) \cap  R(A), $ то
\begin{equation}\label{2}
  u \in H_2(\Omega) \cap C^1(\Omega), \; u_{xx} \in C(\Omega).
\end{equation}

\end{frame}

%%%%%%%%%%%%%%%%%%%%%%%%%%%%%%%%%%%%%%%%%%%%%%%%%%%%%%%%%%%%%%%%%%%%%%%%%%%%%%%%%%%%%%

\section{Нелинейное уравнение}
\subsection{Квазилинейное уравнение с условием  нерезонансности}

\begin{frame}{Квазилинейное уравнение с условием  нерезонансности. Теорема 1.}

\textbf{Теорема 1}.
Пусть выполнены условия (\ref{tabc}), (\ref{a}). Положим, что $g(x,t,u)$ удовлетворяет условиям:
\begin{enumerate}
  \item $g \in C^1(\Omega \times \textbf{R}), \; T-$периодична по $t$;
  \item Существуют константы $\alpha, \beta, \widetilde{u}$, что
  \begin{equation}\label{enenf}
    \alpha \leqslant \frac{g(x,t,u)}{u} \leqslant \beta \;\;\; \forall (x,t) \in \Omega, \;\; u \in (-\infty; -\widetilde{u}] \cup [\widetilde{u}; +\infty),
  \end{equation}
\end{enumerate}
причем
\begin{equation}\label{con}
  \alpha < \beta, \widetilde{u} > 0, [\alpha,\beta] \cap \sigma = \emptyset,
\end{equation}
где $\sigma = sp(A)$. Тогда исходная задача  имеет обобщенное решение $u \in H_2(\Omega)\cap C^1(\Omega), \; u_{xx} \in C(\Omega)$ для любой $T$ --  периодической по $t$ правой части $f \in H_1(\Omega)$. Если дополнительно условиям теоремы функция $g(x,t,u)$ удовлетворяет условию
\begin{equation}\label{42}
  \alpha \leqslant g'_u(x,t,u) \leqslant \beta \;\; \; \forall (u,x,t) \in \Omega \times \textbf{R},
\end{equation}
тогда  задача  имеет единственное решение.

\end{frame}

%%%%%%%%%%%%%%%%%%%%%%%%%%%%%%%%%%%%%%%%%%%%%%%%%%%%%%%%%%%%%%%%%%%%%%%%%%%%%%%%%%%%%%


\subsection{Квазилинейное уравнение с условием  резонансности}

\begin{frame}{Квазилинейное уравнение с условием  резонансности.}

Запишем уравнение в следующем виде:
\begin{equation}\label{54}
u_{tt} + u_{xxxx} - a u_{xx} = g(u) + f(x,t), \;\; x \in (0,\pi),\; t \in \textbf{R}
\end{equation}
Где $g(u):$
\begin{equation}\label{42}
g(u) = \lambda u  + p(u),
\end{equation}
Причем $\lambda \in \sigma$, а функция $p(u)$ ограниченная, то есть существует такая $L$, что
\begin{equation}\label{43}
  |p(u)| \leqslant L \;\; \forall u \in \textbf{R}
\end{equation}
Обозначим $ N_2 = ker (A - \lambda I), \; N_1 = N_2^{\bot}, \; N_1, N_2 \in L_2(\Omega)$. Представим функцию $f \in L_2(\Omega)$ в виде суммы
\begin{equation}\label{44}
  f = f_1 + f_2, \; f_1 \in N_1, \; f_2 \in N_2.
\end{equation}

\end{frame}

%%%%%%%%%%%%%%%%%%%%%%%%%%%%%%%%%%%%%%%%%%%%%%%%%%%%%%%%%%%%%%%%%%%%%%%%%%%%%%%%%%%%%%

\begin{frame}{Квазилинейное уравнение с условием  резонансности. Теорема 2.}

\textbf{Теорема 2}. Пусть выполнены условия (\ref{tabc}), (\ref{a}), (\ref{42}), (\ref{43}), $\lambda \in \sigma$, функция $p(u) \in C^{1}(\textbf{R})$ не убывает или не возрастает по $u$. Также предположим, что функция $f(x,t) \in H_1(\Omega)$ имеет представление (\ref{44}) и слагаемое $f_2$ удовлетворяет либо неравенству:
\begin{equation}\label{461}
  p(-\infty) + \delta \leqslant -f_2(x,t) \leqslant p(+\infty) - \delta \;\; \forall (x,t) \in \Omega
\end{equation}
в том случае, если $p(u)$ не убывает,  либо неравенству
\begin{equation}\label{462}
  p(+\infty) + \delta \leqslant -f_2(x,t) \leqslant p(-\infty) - \delta \;\; \forall (x,t) \in \Omega
\end{equation}
в том случае, если $p(u)$ не возрастает с константой $ \delta > 0$, $(p(\pm\infty) = \lim\limits_{x \rightarrow \pm \infty}p(u))$.  Тогда задача  имеет обобщенное решение  $u \in H_2(\Omega) \cap C^{1}(\Omega)$ и $u_{xx} \in C(\Omega)$

\end{frame}

%%%%%%%%%%%%%%%%%%%%%%%%%%%%%%%%%%%%%%%%%%%%%%%%%%%%%%%%%%%%%%%%%%%%%%%%%%%%%%%%%%%%%%%%%%%%%%%%%%%%%%%%%%%%%%%%%%%%%%%%%%%%%%%%%%%%%%%%%
\section{Постановка задачи 2}
\begin{frame}{Постановка задачи 2.}
\setbeamercovered{transparent}

Рассматривается уравнение Эйлера---Бернулли

\begin{equation}\label{equationkv2}
  u_{tt} + u_{xxxx} - au_{xx} = g(x, t)|u|^{p-2}u + f(x,t), \; x \in (0,\pi), \; t \in \textbf{R}
\end{equation}

с однородными граничными условиями на отрезке

\begin{equation}\label{ickv21}
  u(0,t) = u_{xx}(0,t) = 0, \; t \in \textbf{R}
\end{equation}

\begin{equation}\label{ickv22}
  u_x(\pi,t) = u_{xxx}(\pi,t) - h u(\pi t   ) = 0, \; t \in \textbf{R}
\end{equation}

и с условием периодичности по времени

\begin{equation}\label{periodkv2}
 u(x,t + T) = u(x,t), \; x \in (0,\pi), \; t \in \textbf{R},
\end{equation}
константы $a, h$ являются положительными, а $p$ удовлетворяет условию
\begin{equation}\label{pcond}
  p > 2
\end{equation}


\end{frame}

%%%%%%%%%%%%%%%%%%%%%%%%%%%%%%%%%%%%%%%%%%%%%%%%%%%%%%%%%%%%%%%%%%%%%%%%%%%%%%%%%%%%%%
\section{Линейное уравнение}
\subsection{Линейная задача}
\begin{frame}{Линейная задача}

Задача Штурма---Лиувилля примет следующий вид

\begin{equation}\label{task2}
  Y'''' - a Y'' = \lambda Y, x \in (0,\pi),\; a > 0
\end{equation}

с граничными условиями

\begin{gather*}
  Y(0) = Y''(0) = 0  \label{bc21}, \\
  Y'(\pi) = Y'''(\pi) - h Y(\pi) = 0 \label{bc22}.
\end{gather*}

Трансцендентное уравнение:

\begin{multline}\label{tre2}
 \frac{1}{\ctg\bigg(\pi \sqrt{\frac{1}{2}(\sqrt{a^2 + 4 \lambda} - a)}\bigg)} = \\
 =\sqrt{\frac{ \sqrt{a^2 + 4 \lambda} - a}{\sqrt{a^2 + 4 \lambda} + a}}\th\bigg(\pi \sqrt{\frac{1}{2}(\sqrt{a^2 + 4 \lambda} + a)}\bigg) - \\
 - \frac{1}{h}\bigg(2 \sqrt[\frac{2}{3}]{\frac{1}{2}(\sqrt{a^2 + 4 \lambda} - a)} + \sqrt{\frac{1}{2}(\sqrt{a^2 + 4 \lambda} - a)} a\bigg)
\end{multline}

\end{frame}

%%%%%%%%%%%%%%%%%%%%%%%%%%%%%%%%%%%%%%%%%%%%%%%%%%%%%%%%%%%%%%%%%%%%%%%%%%%%%%%%%%%%%%

\begin{frame}

Собственные функции можно представить ввиде:

\begin{equation}
Y_n = C_n (\sin(b_n x) -  \sh(c_n x) \frac{b_n \cos(b_n \pi)}{c_n \ch(c_n \pi)}).
\end{equation}

После локализации собственных значений, получим
\begin{equation}\label{ln}
  \lambda_n = \bigg(n - \frac{1}{2} + \rho_n\bigg)^4 + a\bigg(n - \frac{1}{2} + \rho_n\bigg)^2,\;\rho_n \in \bigg(0;\frac{1}{2}\bigg)
\end{equation}

где величина $\rho_n$ обладает следующим свойством
\begin{equation}\label{thet}
 C_1 \cdot \frac{1}{n^3} < \theta_n < C_2 \cdot \frac{1}{n^3},\; C_1, C_2 \in \textbf{R},\;n \in \textbf{N}
\end{equation}

\end{frame}

%%%%%%%%%%%%%%%%%%%%%%%%%%%%%%%%%%%%%%%%%%%%%%%%%%%%%%%%%%%%%%%%%%%%%%%%%%%%%%%%%%%%%%

\section{Нелинейное уравнение}
\subsection{Лемма}

\begin{frame}{Лемма.}

Введем следующие условия
\begin{equation}\label{cond3}
   T = 2\pi\frac{q}{p}, \; \text{НОД}(q, p) = 1
\end{equation}
\begin{equation}\label{cond4}
   a > 0, \;\; (2a + 1)q/4 \ne N
\end{equation}
Оператор исходной задачи $P$ представим в виде
\begin{equation}
  P = \frac{\partial^2}{\partial t^2} + \frac{\partial^4}{\partial x^4} - a \frac{\partial^2}{\partial x^2}
\end{equation}
\textbf{Лемма. }\textit{Пусть выполнены условия (\ref{cond3}), (\ref{cond4}), тогда ядро $N(P)$ оператора P является конечномерным, а оператор $P^{-1}: R(D) \rightarrow R(D)$ является вполне непрервным}\\

\end{frame}

%%%%%%%%%%%%%%%%%%%%%%%%%%%%%%%%%%%%%%%%%%%%%%%%%%%%%%%%%%%%%%%%%%%%%%%%%%%%%%%%%%%%%%

\subsection{Теорема}

\begin{frame}{Теорема.}
Будем предполагать выполнение следующих условий:
\begin{equation}\label{cond1}
  g(x, t) \in C^1(\Omega), \text{а также } T-\text{периодична по } t,
\end{equation}
\begin{equation}\label{cond2}
  g(x, t) > 0 \;\; \forall(x, t) \in \Omega, 
\end{equation}

тогда сформулируем теорему, которая будет доказана в этой работе
\\
\textbf{Теорема. } Пусть выполнены условия (\ref{pcond}), (\ref{cond3}) --- (\ref{cond2}), тогда для любого $d > 0$ задача (\ref{equationkv2}) --- (\ref{periodkv2}) имеет обобщенное решение \\
\begin{equation}
  u \in H_2(\Omega) \cap C^1(\Omega),
\end{equation}
такое, что $\|u\|_p \geqslant d,$ а $u_{xx}, u_{xxx} \in C(\Omega)$ и граничные условия (\ref{ickv21}), (\ref{ickv22}) выполнены в классическом смысле
\end{frame}

%%%%%%%%%%%%%%%%%%%%%%%%%%%%%%%%%%%%%%%%%%%%%%%%%%%%%%%%%%%%%%%%%%%%%%%%%%%%%%%%%%%%%%

\section{Заключение}

\begin{frame}{Заключение.}

\begin{enumerate}
  \item Исследована соответствующая задача Штурма-Лиувилля на поиск собственных значений и собственных функций.
  \item Получено трансцендентное уравнение на собственные значения и проведено исследование его корней.
  \item Произведена полная локализация собственных значений.
  \item Выведены асимптотические оценки для собственных значений.
  \item Доказаны теоремы о существовании и единственности решения исходной задачи в нерезонансном и резонансном случаях.
\end{enumerate}

\end{frame}

%%%%%%%%%%%%%%%%%%%%%%%%%%%%%%%%%%%%%%%%%%%%%%%%%%%%%%%%%%%%%%%%%%%%%%%%%%%%%%%%%%%%%%%%%%%%%%%%%%%%%%%%%%%%%%%%%%%%%%%%%%%%%%%%%%%%%%%%%


\section{Список литературы}

\begin{frame}

\begin{thebibliography}{9}

\bibitem{Yama}Yamaguchi M., Existence of periodic solutions of second order nonlinear evolution equations and applications. Funkcialaj Ekvacioj. 1995. v. 38. p. 519-538.
\bibitem{JiS}Ji S. Periodic solutions for one dimensional wave equation with bounded nonlinearity. J. Differential Equations. 2018. v. 264. No 9. p. 5527-5540.
\bibitem{Tihonov} А.Н. Тихонов, А.А. Самарский. Уравнения математической физики. М: Издательство Наука, 1977. -  735 с.
\bibitem{Rudakov3}Рудаков И.А. Задача о периодических колебаниях двутавровой балки с  жестко закрепленным концом в случае резонанса. Дифференциальные уравнения, 2020. -  с. 691-700.
\bibitem{Rudakov5}Рудаков И.А. Периодические решения квазилинейного уравнения Эйлера– Бернулли. Дифференциальные уравнения, 2019. - c. 1581-1583.
\bibitem{Rudakov6}Рудаков И.А. Уравнение колебаний балки с закрепленными и шарнирно опертыми концами. Вестник МГУ. Серия 1. Математика. Механика. 2020. №2 - с. 3-8.


\end{thebibliography}

\end{frame}

%%%%%%%%%%%%%%%%%%%%%%%%%%%%%%%%%%%%%%%%%%%%%%%%%%%%%%%%%%%%%%%%%%%%%%%%%%%%%%%%%%%%%%%%%%%%%%%%%%%%%%%%%%%%%%%%%%%%%%%%%%%%%%%%%%%%%%%%%

\end{document}
