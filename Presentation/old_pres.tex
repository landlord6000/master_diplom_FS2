%\documentclass[a4paper]{article}
%\usepackage{beamerarticle}
\documentclass[ignorenonframetext,unicode,handout, 9pt]{beamer}

\usepackage[utf8]{inputenc}
\usepackage[T2A]{fontenc}
\usepackage[english,russian]{babel}
\usepackage{amsmath}
\usepackage{amsfonts}
\usepackage{amssymb}
\usepackage{graphicx,pgf}
\usepackage{multimedia}
%\usepackage{hyperref}

%\usetheme{Rochester}  %тема без навигации
%\usetheme{Montpellier} %тема с навигацией в виде дерева
%\usetheme{Berkeley} %тема с оглавлением на полях
%\usetheme{Berlin} %тема с навигацией в виде мини-слайдов
\mode<presentation>{
\usetheme{Copenhagen} %тема с таблицей разделов и подразделов

\graphicspath{{pictures/}}
\DeclareGraphicsExtensions{.pdf,.png,.jpg}

%\useinnertheme{circles}   %внутренняя тема
%\useoutertheme{smoothbars}   %внешняя тема
\usecolortheme{crane}     %цветовая схема
\usefonttheme{serif}    %шрифты
}
% \\ {} \and \\  \\[3mm]
\title[Решение жестких задач методом Обрешкова]{Решение жестких задач методом Обрешкова}
\author[Выполнил: М.Д Зиновьев]{Выполнил: М.Д Зиновьев \and  \\ Науч. руководитель: к.т.н. Котович А.В.}
\institute{МГТУ имени Н.Э.Баумана (национальный исследовательский университет) \and Кафедра ФН-2(прикладная математика)}
\date{\today}
\titlegraphic{\includegraphics[width=2cm]{logo.png}}
%\setbeamersize{text margin left=4cm}

\begin{document}



\begin{frame}[plain]
\maketitle
\end{frame}

\section{Постановка задачи}

\begin{frame}{Задача}
\setbeamercovered{transparent}

Исследовать неявный метод Обрешкова для решения жестких задач, сравнить полученное данным методом решение с решением, полученным явным методом. Провести анализ наблюдаемых явлений и
сделать выводы о целесообразности исользования неявных методов для решения жестких задач.

\end{frame}

\section[Определение жесткости]{Определение жесткости}

\begin{frame}[t]{Определение жесткости}

Система ОДУ $u' = Au$ с постоянной матрицей $A = A_{m \times m}$ называется жесткой, если:
\begin{enumerate}
  \item{все собственные числа матрицы A имеют отрицательную действительную часть, т.е. $Re\lambda_i < 0, i = 1,2,..,m$},
  \item{число}
  $$S = \frac{\max\limits_{1 \leqslant k \leqslant m}|Re\lambda_k|}{\min\limits_{1 \leqslant k \leqslant m}|Re\lambda_k|},$$
  называемое числом жесткости, велико.
  \end{enumerate}

  \end{frame}
  \begin{frame}[t]

  В случае задачи Коши для общей нелинейной системы
\begin{equation}\label{1}
 u_t =f(t,u), 0 \leqslant t \leqslant T, u(0) = u_0
\end{equation}
поведение её решения вблизи некоторой точки $(t_0,u_0)$ определяется матрицей Якоби $A:$
$$A = \frac{\partial f}{\partial u} =
\left(
\begin{array}{ccc}
\frac{\partial f_1}{\partial u_1} & \ldots & \frac{\partial f_1}{\partial u_m} \\
\ldots & \ldots & \ldots \\
\frac{\partial f_m}{\partial u_1} & \ldots & \frac{\partial f_m}{\partial u_m} \\
\end{array}
\right).
$$
Система называется жесткой, если для всех $t,u$ (т.е. на решениях (\ref{1})), собственные значения матрицы $A$ удовлетворяют следующим условиям:
$$\frac{\max\limits_i |Re \lambda_i|}{\min\limits_j |Re \lambda_j|} >> 1,$$
$$Re \lambda_i < 0,$$
$$\max_i |Im \lambda_i| <<  \max_j |Re \lambda_j|, i,j = 1,2,...,m.$$

\end{frame}

\section[Схемы Обрешкова]{Схемы Обрешкова}

\begin{frame}[t]{Схемы Обрешкова}
Задача Коши выглядит так:
\begin{equation}
\label{ZK}
 \begin{cases}
   u_t = f(t,u)\\
   0 \leqslant t \leqslant T \\
   u(0) = u_0
 \end{cases}
\end{equation}
Общий вид линейных  многошаговых методов:
\begin{equation}\label{2}
\sum\limits_{k=0}^{K}a_k u_{n+k} - \tau b_k f(t_n + k \tau, u_{n+k}), a_K \ne 0.
\end{equation}
Добавим  в (\ref{2}) линейную комбинацию вторых производных $u_{tt}$ с коэффициентами $c_k$
\begin{equation}\label{3}
\sum\limits_{k=0}^{K}a_k u_{n+k} - \tau b_k f(t_n + k \tau, u_{n+k}) - \tau^2 c_k  u_{tt}(t_n + k \tau, u_{n+k}) = 0.
\end{equation}
Используем  выражение для второй производной:
\begin{equation}\label{VP}
u_{tt} = f_t = \frac{\partial f}{\partial t} + f_u f.
\end{equation}

\end{frame}
\begin{frame}[t]

Получаем

\begin{eqnarray*} \sum\limits_{k=0}^{K}a_k u_{n+k} - \tau b_k f(t_{n+k}, u_{n+k}) - \tau^2 c_k (\partial f(t_{n+k},u_{n+k})/ \partial t + \\
   + f_u(t_{n+k},u_{n+k}) f(t_{n+k},u_{n+k})) = 0.
\end{eqnarray*}


Условия, обеспечивающие $p-$ый порядок аппроксимации имеют вид схожий с условиями для линейных многошаговых методов:
\begin{equation}
\label{UA}
 \begin{cases}
   \sum\limits_{k = 0}^{K} a_k = 0,  \sum\limits_{k = 0}^{K}k a_k - b_k = 0,\\
   \sum\limits_{k = 0}^{K} (k^2 \frac{a_k}{2} - k b_k - c_k) = 0 ,\\
    \ldots \\
     \sum\limits_{k = 0}^{K}( k^p \frac{a_k}{p!} - k^{p-1}\frac{ b_k}{(p-1)!} - k^{p-2} \frac{c_k}{(p-2)!} ) = 0.
 \end{cases}
\end{equation}

\end{frame}

\section{Устойчивость схем Обрешкова}

\begin{frame}[t]{Устойчивость схем Обрешкова}
Из условий порядка (\ref{UA}) следует, что схема с коэффициентами:
$$ a_0 = -1, a_1 = 1, b_0 = \frac{1}{2}, b_1 = \frac{1}{2}, c_0 = \frac{1}{12}, c_1 = \frac{1}{12}$$
даёт четвертый порядок аппроксимации. Воспользуемся модельным уравнением $u' = \lambda u,$ (обозначение $\lambda \tau = \sigma$), тогда получаем \\
$$u_t = f = \lambda u,$$
$$u_{tt} = f_t = \lambda u_t =  \lambda^2 u,$$
$$\sum\limits_{k=0}^{K}(a_k - \sigma b_k - \sigma^2 c_k)u_{n+k} =  \sum\limits_{k=0}^{K}\gamma_{k} u_{n+k} = 0, \qquad   \gamma_k = a_k - \sigma b_k - \sigma^2 c_k.$$
$$u_{n+1} = q u_{n},$$
$$q = -\frac{\gamma_0}{\gamma_1} = \frac{1 + \sigma(1/2 + c_0 + c_1) + \sigma^2 c_0}{1 - \sigma(1/2 - c_0 - c_1) - \sigma^2 c_1}.$$

\end{frame}

\begin{frame}[t]
\begin{figure}[htbp]
\center{\includegraphics[scale=0.27]{Ris1}}
\text{Рис.1. Схемы Обрешкова в плоскости неопределённых коэффициентов (K=1)}
\label{fig:image}
\end{figure}
\end{frame}


\section[Простейшая модель гликолиза]{Простейшая модель гликолиза}

\begin{frame}[t]{Простейшая модель гликолиза}
Рассмотрим простейшую модель гликолиза, предложенную Дж. Хиггинсом
\begin{equation}
\label{GL}
 \begin{cases}
   \dot{y}_1 = 1 - y_1 y_2, \\
   \dot{y}_2 = \alpha y_2 \bigg( y_1 = \frac{1 + \beta}{y_2 + \beta} \bigg), \\
   y_1(0) = 1, \qquad y_2(0) = 0.001, \qquad T_k = 50, \\
   \beta = 10, \qquad \alpha = 100.
 \end{cases}
\end{equation}
\end{frame}


\begin{frame}[t]
\begin{figure}[htbp]
\center{\includegraphics[scale=0.4]{RK4}}
\text{Рис.2. Фазовая траектория, полученная при решении методом Рунге---Кутты $4$-ого порядка}
\label{fig:image}
\end{figure}
\end{frame}

\begin{frame}[t]
\begin{figure}[htbp]
\center{\includegraphics[scale=0.38]{FT}}
\text{Рис.3. Фазовая траектория, полученная при решении методом Обрешкова $4$-ого порядка}
\label{fig:image}
\end{figure}
\end{frame}

\begin{frame}[t]
\begin{figure}[htbp]
\center{\includegraphics[scale=0.38]{PC}}
\text{Рис.4. Предельный цикл фазовых траекторий, полученный при решении методом Обрешкова $4$-ого порядка}
\label{fig:image}
\end{figure}
\end{frame}

\section{Вывод}
\begin{frame}{Вывод}
В работе была рассмотрена простейшая модель гликолиза, было показано, что эта задача является жесткой и требует использования неявных методов. Был рассмотрен и исследован метод Обрешкова. Было выяснено, что данный класс схем более обширный, чем класс линейных многошаговых схем и позволяет достигнуть более высокой точности на меньшем числе точек шаблона.
\end{frame}
\end{document}
